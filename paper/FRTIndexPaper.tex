%%%%%%%%%%%%%%%%%%%%%%%%%%%
% Explanation of the motivation and construction of the FRT Index
% Christopher Gandrud
% 19 February 2014
%%%%%%%%%%%%%%%%%%%%%%%%%%%%


\documentclass[a4paper]{article}
\usepackage{fullpage}
\usepackage[authoryear]{natbib}
\usepackage{setspace}
    \doublespacing
%\usepackage[usenames,dvipsnames]{xcolor}
\usepackage{hyperref}
\hypersetup{
    colorlinks,
    citecolor=black,
    filecolor=black,
    linkcolor=cyan,
    urlcolor=cyan
}
\usepackage{dcolumn}
\usepackage{booktabs}
\usepackage{url}
\usepackage{tikz}
\usepackage{todonotes}
\usepackage[utf8]{inputenc} 
\usepackage{graphicx}

%%%%%%%%% Title
\title{Measuring Financial Regulatory Transparency}

\author{Christopher Gandrud \\ 
    {\emph{Hertie School of Governance}}\footnote{Friedrichstra{\ss}e 180. 10117 Berlin, Germany. Contact email: \href{mailto:gandrud@hertie-school.org}{gandrud@hertie-school.org}.}}


\begin{document}

\maketitle

%%%%%%%%% Abstract
\begin{abstract}
\noindent \emph{Early working draft. Comments welcome.} \\
For financial supervision to be effective, supervisors need accurate information about financial sector activities. For financial supervisors to be accountable  
\end{abstract}


\subsection{Data}

To measure financial supervisory transparency we first gathered data on whether or governments reported data on a number of indicators included in the World Bank's Global Financial Development Data set first created by \cite{Cihak2012}.\footnote{Access to the most updated version of the data set is available through \url{http://data.worldbank.org/data-catalog/global-financial-development}. Accessed February 2014.} We followed Hollyer et al.`s \citeyearpar{Hollyer2014} criteria for inclusion of variables and countries. First, we only include indicators that are reported by at least one country for each year in the period 1998-2011. This gave us the greatest coverage of indicators that are comparable across countries. Second, we excluded all indicators that were explicitly gathered for only a subset of countries. Third, we did not include any indicator that was primarily from a non-governmental source. This included both indicators from World Bank Sponsored surveys, such as the Global Financial Inclusion Survey and the Enterprise Survey. It also included data primarily derived from sources such as Swiss Re's Sigma Reports, Standard \& Poor, Bankscope, and Bloomberg. Fourth, we did not include variables that are linear combinations of other variables. Fifth, we did not include variables that were simply references to the same quantity in different units. [CHECK TO SEE IF 4 AND 5 ARE RELEVANT] Sixth, we excluded small countries with populations less than 500,000 [Is this appropriate? Maybe this excludes some offshore centers?]    



\bibliographystyle{apsr}
\bibliography{FRTIndex}

\end{document}