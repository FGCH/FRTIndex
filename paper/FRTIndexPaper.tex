%%%%%%%%%%%%%%%%%%%%%%%%%%%
% Explanation of the motivation and construction of the FRT Index
% Christopher Gandrud
% 20 February 2014
%%%%%%%%%%%%%%%%%%%%%%%%%%%%


\documentclass[a4paper]{article}
\usepackage{fullpage}
\usepackage[authoryear]{natbib}
\usepackage{setspace}
    \doublespacing
%\usepackage[usenames,dvipsnames]{xcolor}
\usepackage{hyperref}
\hypersetup{
    colorlinks,
    citecolor=black,
    filecolor=black,
    linkcolor=cyan,
    urlcolor=cyan
}
\usepackage{dcolumn}
\usepackage{booktabs}
\usepackage{url}
\usepackage{tikz}
\usepackage{todonotes}
\usepackage[utf8]{inputenc} 
\usepackage{graphicx}
\usepackage{longtable}

%%%%%%%%% Title
\title{Measuring Financial Regulatory Transparency}

\author{Mark Copelovich \\ \emph{University of Wisconsin, Madison} \\[0.5cm] Christopher Gandrud and Mark Hallerberg \\ 
    {\emph{Hertie School of Governance}}\footnote{Friedrichstra{\ss}e 180. 10117 Berlin, Germany. Contact email: \href{mailto:gandrud@hertie-school.org}{gandrud@hertie-school.org}. All material for replicating the FRT Index and the analysis in this paper can be found at: \url{https://github.com/FGCH/FRTIndex}.}}


\begin{document}

\maketitle

%%%%%%%%% Abstract
\begin{abstract}
\noindent \emph{Early working draft. Comments welcome.} \\
For financial supervision to be effective, regulators need have accurate information about financial sector activities. For the public to be able to hold supervisors accountable then need access to the information financial supervisors have about the health of the banking system. In this paper we use Bayesian item response theory techniques to create a global and comparable Financial Regulatory Transparency ((FRT) Index.  
\end{abstract}

[INTRODUCTION]

In previous research we have found that even within the relatively homogeneous European Union with supranational authorities tasked with gathering and reporting aggregate financial data from member states there is considerable variation in what is actually reported \cite[see][]{Gandrud2014a}.  

\section{Creating the FRT Index}

We treat financial regulatory transparency as an unobserved latent variable that effectively summarizes countries likelihood of reporting yearly data that is included in the World Bank's Global Financial Development data set first created by \cite{Cihak2012}.\footnote{Access to the most updated version of the data set is available through \url{http://data.worldbank.org/data-catalog/global-financial-development} Accessed February 2014.}

\subsection{Included indicators}

To measure financial supervisory transparency we first gathered data on whether or governments reported data on a number of indicators included in the World Bank's Global Financial Development data set. For a full list of the indicators included, please see the Supplementary Materials. We followed Hollyer et al.`s \citeyearpar{Hollyer2014} criteria for inclusion of variables and countries. First, we only include indicators that are reported by at least one country for each year in the period 1998-2011. This gave us the greatest coverage of indicators that are comparable across countries. Second, we excluded all indicators that were explicitly gathered for only a subset of countries. As such we avoided including data where the primary source was the Bank for International Settlements. Third, we did not include any indicator that was primarily from a non-governmental source. This included both indicators from World Bank Sponsored surveys, such as the Global Financial Inclusion Survey and the Enterprise Survey. It also included data primarily derived from sources such as Swiss Re's Sigma Reports, Standard \& Poor, Bankscope, and Bloomberg. Fourth, we did not include variables that are linear combinations of other variables. Fifth, we did not include variables that were simply references to the same quantity in different units. [CHECK TO SEE IF 4 AND 5 ARE RELEVANT] Sixth, we excluded small countries with populations less than 500,000. [Is this appropriate? Maybe this excludes some offshore centers?] 

In addition we did not include countries with gross domestic products per capita of less than 200 US dollars.\footnote{The population and GDP per capita data was gathered from the World Bank's development indicators \citeyearpar{WDIMain}} Countries with levels of income this low likely do not have financial systems sophisticated enough to have the quantities reported in the indicators. 

\section{Validity}

[OTHER TRANSPARENCY INDICATORS TO COMPARE AGAINST?]

\subsection{The FRT Index}

\subsection{Indicator discrimination}

\section{Preliminary Associations}

To demonstrate the potential usefulness of the FRT Index we examine a number of associations between the Index and the occurrence and potential occurrence of financial crisis.

[ASSOCIATION WITH ECONOMIC BUREAUCRATIC CAPACITY]
[Z-SCORE (PROB. OF BANK DEFAULT) AS DEPENDENT VARIABLE]

\bibliographystyle{apsr}
\bibliography{FRTIndex}

\section*{Supplementary Materials}

\begin{table}[h]
    \caption{Indicators included in the FRT Index from the World Bank's Global Financial Development data set}
    \label{IndTable}
    \vspace{0.3cm}
        % latex table generated in R 3.0.2 by xtable 1.7-1 package
% Thu Feb 20 10:58:09 2014
{\scriptsize
\begin{tabular}{llll}
  \hline
SeriesCode & Indicator.Name & Source & Periodicity \\ 
  \hline
GFDD.DI.01 & Private credit by deposit money banks to GDP (\%) & IFS/IMF & 1961-2011 \\ 
  GFDD.DI.02 & Deposit money banks' assets to GDP (\%) & IFS/IMF & 1961-2011 \\ 
  GFDD.DI.03 & Nonbank financial institutions’ assets to GDP (\%) & IFS/IMF & 1961-2011 \\ 
  GFDD.DI.04 & Deposit money bank assets to deposit money bank assets and central bank assets (\%) & IFS/IMF & 1960-2011 \\ 
  GFDD.DI.05 & Liquid liabilities to GDP (\%) & IFS/IMF & 1961-2011 \\ 
  GFDD.DI.06 & Central bank assets to GDP (\%) & IFS/IMF & 1961-2011 \\ 
  GFDD.DI.07 & Mutual fund assets to GDP (\%) & World Bank & 1980-2011 \\ 
  GFDD.DI.08 & Financial system deposits to GDP (\%) & IFS/IMF & 1961-2011 \\ 
  GFDD.DI.11 & Insurance company assets to GDP (\%) & World Bank & 1980-2011 \\ 
  GFDD.DI.12 & Private credit by deposit money banks and other financial institutions to GDP (\%) & IFS/IMF & 1961-2011 \\ 
  GFDD.DI.13 & Pension fund assets to GDP (\%) & World Bank & 1990-2011 \\ 
  GFDD.DI.14 & Domestic credit to private sector (\% of GDP) & World Bank & Annual: \\ 
  GFDD.EI.02 & Bank lending-deposit spread & IFS/IMF & 1980-2011 \\ 
  GFDD.EI.08 & Credit to government and state owned enterprises to GDP (\%) & IFS/IMF & 1980-2011 \\ 
  GFDD.OI.02 & Bank deposits to GDP (\%) & IFS/IMF & 1961-2011 \\ 
  GFDD.OI.07 & Liquid liabilities in millions USD (2000 constant) & IFS/IMF & 1960-2011 \\ 
  GFDD.OI.13 & Remittance inflows to GDP (\%) & World Bank & 1970-2011 \\ 
  GFDD.SI.02 & Bank nonperforming loans to gross loans (\%) & IFSI/IMF & 1998-2011 \\ 
  GFDD.SI.03 & Bank capital to total assets (\%) & IFSI/IMF & 1998-2011 \\ 
  GFDD.SI.04 & Bank credit to bank deposits (\%) & IFS/IMF & 1960-2011 \\ 
  GFDD.SI.05 & Bank regulatory capital to risk-weighted assets (\%) & IFSI/IMF & 1998-2011 \\ 
  GFDD.SI.07 & Provisions to nonperforming loans (\%) & IFSI/IMF & 1998-2011 \\ 
   \hline
\end{tabular}
}

    }
\end{table}


\end{document}