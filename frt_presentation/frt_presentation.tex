\documentclass{beamer}
\usetheme{Stats}
\setbeamercovered{transparent}
\usepackage{color}
\usepackage{hyperref}
  \hypersetup{
      colorlinks=true
        linkcolor=black
        }
\usepackage{url}
\usepackage{graphics}
\usepackage{tikz}
\usepackage{booktabs}


%%%%%%%%%%%%%%%%%%%%%%%%%%%%%%%% Title Slide %%%%%%%%%%%%%%%%%%%%%%%%%%
\title[]{Measuring International Financial Supervisory Transparency}
\author[]{
    \href{mailto:gandrud@hertie-school.org}{Christopher Gandrud}, Mark Copelovitch, and Mark Hallerberg
}
\date{\today}

\begin{document}

\frame{\titlepage}

\section{Motivation}
    \frame{
        \frametitle{Why financial supervisory transparency?}
        Financial supervisory transparency has been \textbf{lauded} as promoting:

        \begin{itemize}
                \item financial system stability,
        \item democratic legitimacy for supervisors.
        \end{itemize}

    }

    \frame{
        \frametitle{Promotion}
            Supervisory transparency has been \textbf{promoted} by international/supra-national institutions including the IMF, Basel Committee, and the European Union.
    }

    \frame{
        \frametitle{But\ldots}

    We \textbf{{\large{lack}} reliable}, \textbf{cross-country}, and \textbf{cross-time} indicators of financial supervisory transparency to \textbf{{\large{test}}} these assertions.
    }

    \frame{
        \frametitle{Objective}
        Our objectives are to:
        \begin{itemize}
            \item \textbf{Develop} a reliable and valid indicator of supervisory transparency across countries and time.
                \begin{itemize}
                    \item Largely complete.
                \end{itemize}
            \item Use this to \textbf{examine}:
                \begin{itemize}
                    \item \textbf{why} countries become more/less transparent,
                    \item \textbf{how}, if at all supervisory transparency affects economic outcomes.
                \end{itemize}
        \end{itemize}
    }

    \frame{
        \frametitle{Methodological Contribution}
        We make (at least) two important methodological contributions:
        \begin{itemize}
            \item Develop a Bayesian Item Response Theory-based \textbf{unique indicator} of countries' willingness to reveal basic facts about their financial systems to international actors.
            \item Show that \textbf{missing financial system data} is \textbf{often endogenous} to financial system difficulties and policymaker's aspirations.
        \end{itemize}
    }

    \section{Creating the FRT Index}
    \frame{
        \frametitle{Predecessors}

    }


\end{document}
